\documentclass[ignorenonframetext,]{beamer}
\usetheme{Warsaw}
\usepackage{amssymb,amsmath}
\usepackage{ifxetex,ifluatex}
\ifxetex
  \usepackage{fontspec,xltxtra,xunicode}
  \defaultfontfeatures{Mapping=tex-text,Scale=MatchLowercase}
\else
  \ifluatex
    \usepackage{fontspec}
    \defaultfontfeatures{Mapping=tex-text,Scale=MatchLowercase}
  \else
    \usepackage[utf8x]{inputenc}
  \fi
\fi
% Comment these out if you don't want a slide with just the
% part/section/subsection/subsubsection title:
\AtBeginPart{\frame{\partpage}}
\AtBeginSection{\frame{\sectionpage}}
\AtBeginSubsection{\frame{\subsectionpage}}
\AtBeginSubsubsection{\frame{\subsubsectionpage}}
\setlength{\parindent}{0pt}
\setlength{\parskip}{6pt plus 2pt minus 1pt}
\setlength{\emergencystretch}{3em}  % prevent overfull lines
\setcounter{secnumdepth}{0}

\title{Node JS : Introduction to Server-Side Javascript}
\author{Andrew Gall}
\date{November 9, 2012}

\begin{document}
\frame{\titlepage}

\section{Introduction}

\begin{frame}\frametitle{My Experience}

\begin{itemize}
\item
  Drexel University
\item
  Worked for a start-up called CoupedOut for the past two years
\item
  Work totally remotely
\end{itemize}
\end{frame}

\begin{frame}\frametitle{Your Experience}

\begin{itemize}
\item
  Java?
\item
  Python?
\item
  Javascript?
\item
  JQuery?
\end{itemize}
\end{frame}

\begin{frame}\frametitle{Follow Along}

\begin{itemize}
\item
  Presentation:
  \href{http://housepage.org/nodejs.html}{http://housepage.org/nodejs.html}
\item
  Code:
  \href{http://github.com/housepage/nodejs-tutorial}{http://github.com/housepage/nodejs-tutorial}
\end{itemize}
\end{frame}

\section{Javascript}

\begin{frame}\frametitle{What have you done?}

\begin{itemize}[<+->]
\item
  What is it used for?
\item
  What have you built with it?
\end{itemize}
\end{frame}

\begin{frame}\frametitle{What is
\href{http://en.wikipedia.org/wiki/JavaScript}{Javascript}?}

\begin{itemize}[<+->]
\item
  \href{http://en.wikipedia.org/wiki/Scripting\_language}{Scripting
  Language}
\item
  \href{http://en.wikipedia.org/wiki/Prototype-based}{Prototype-based}
\item
  \href{http://en.wikipedia.org/wiki/Dynamic\_language}{Dynamic} and
  \href{http://en.wikipedia.org/wiki/Weak\_typing}{weakly typed}
\item
  \href{http://en.wikipedia.org/wiki/First-class\_functions}{First-class
  functions}
\end{itemize}
\end{frame}

\begin{frame}\frametitle{Hello World}

\begin{itemize}
\item
  Just writes to standard out
\item
  File: helloworld.js
\end{itemize}
\begin{quote}
console.log(``Hello World'');

\end{quote}
\begin{itemize}
\item
  Run It:
\end{itemize}
\begin{quote}
node helloworld.js

\end{quote}
\end{frame}

\section{Node.JS}

\begin{frame}\frametitle{What is
\href{http://en.wikipedia.org/wiki/Nodejs}{Node JS}?}

\begin{itemize}
\item
  Server side software system designed for writing scalable Internet
  applications, notably web servers.
\item
  Written in Javascript using event-driven, asynchronous I/O to minimize
  overhead and maximize scalability.
\end{itemize}
\end{frame}

\begin{frame}\frametitle{A Sample Application in Node JS}

Let's keep it simple, but realistic:

\begin{itemize}
\item
  The user should be able to use our application with a web browser
\item
  The user should see a welcome page when requesting http://domain/start
  which displays a file upload form
\item
  By choosing an image file to upload and submitting the form, this
  image should then be uploaded to http://domain/upload, where it is
  displayed once the upload is finished
\end{itemize}
\end{frame}

\begin{frame}\frametitle{A Couple of Key Parts}

\begin{itemize}
\item
  An HTTP Server - We want to serve web pages, therefore we need an HTTP
  server.
\item
  A Router - Our server will need to answer differently to requests,
  depending on which URL the request was asking for, thus we need some
  kind of router in order to map requests to request handlers
\item
  Request Handlers - To fullfill the requests that arrived at the server
  and have been routed using the router, we need actual request handlers
\item
  Request Data Handling - The router probably should also treat any
  incoming POST data and give it to the request handlers in a convenient
  form, thus we need request data handling
\item
  We not only want to handle requests for URLs, we also want to display
  content when these URLs are requested, which means we need some kind
  of view logic the request handlers can use in order to send content to
  the user's browser
\item
  Upload Handling - Last but not least, the user will be able to upload
  images, so we are going to need some kind of upload handling which
  takes care of the details
\end{itemize}
\end{frame}

\begin{frame}[fragile]\frametitle{Basic HTTP Server}

\begin{verbatim}
var http = require("http");

http.createServer(function(request, response) {
   response.writeHead(200, {"Content-Type": "text/plain"});
   response.write("Hello World");
   response.end();
}).listen(8888);
\end{verbatim}
\end{frame}

\begin{frame}[fragile]\frametitle{Break It Down}

\begin{verbatim}
var http = require("http");
\end{verbatim}
The first line requires the http module that ships with Node.js and
makes it accessible through the variable http.

\end{frame}

\begin{frame}[fragile]\frametitle{Break It Down}

\begin{verbatim}
var server = http.createServer();
server.listen(8888); 
\end{verbatim}
This part creates the server and makes it listen on port 8888. The
server currently does nothing.

\end{frame}

\begin{frame}[fragile]\frametitle{Break It Down}

\begin{verbatim}
http.createServer(function(request, response) {
   response.writeHead(200, {"Content-Type": "text/plain"});
   response.write("Hello World");
   response.end();
}).listen(8888);
\end{verbatim}
We pass a function into the server which it uses to handle all incoming
requests. We make changes to the response object and then these are
transmitted when we call:

response.end();

\end{frame}

\section{Asynchronous Callbacks}

\begin{frame}[fragile]\frametitle{Why do we need them?}

\begin{verbatim}
var result = database.query("SELECT * FROM hugetable");
console.log("Hello World");
\end{verbatim}
\begin{itemize}
\item
  Database query might take a long time
\item
  Node.JS runs in an event loop
\item
  Important characteristics of how things work in Node.JS
\item
  If you want to know more, check out
  \href{http://debuggable.com/posts/understanding-node-js:4bd98440-45e4-4a9a-8ef7-0f7ecbdd56cb}{Felix
  Geisendörfer's excellent post on node.js}
\end{itemize}
\end{frame}

\begin{frame}[fragile]\frametitle{It becomes this}

\begin{verbatim}
database.query("SELECT * FROM hugetable", function(rows) {
  var result = rows;
});
console.log("Hello World");
\end{verbatim}
\begin{itemize}
\item
  Allows us to do other things while we wait for our results
\end{itemize}
\end{frame}

\section{Organization}

\begin{frame}[fragile]\frametitle{Organizing our code into modules}

\begin{verbatim}
var http = require("http");

function start() {
  function onRequest(request, response) {
    /* ... */
  }

  http.createServer(onRequest).listen(8888);
  console.log("Server has started.");
}

exports.start = start;
\end{verbatim}
We just need to add the line seen at the bottom here.

\end{frame}

\begin{frame}[fragile]\frametitle{Using our new module}

Now this file can be imported elsewhere and will expose that function or
property. So we can do:

\begin{verbatim}
var server = require('./server');
server.start()
\end{verbatim}
\end{frame}

\section{Routing}

\begin{frame}[fragile]\frametitle{Making the Router}

We start simply and just do a print in our router.

\begin{verbatim}
function route(pathname) {
  console.log("About to route a request for " + pathname);
}

exports.route = route; 
\end{verbatim}
\end{frame}

\begin{frame}[fragile]\frametitle{Wiring the Router In}

Now, we add this routing to our HTTP Server:

\begin{verbatim}
function start(route) {
  function onRequest(request, response) {
    /* ... */

    route(pathname);

    response.writeHead(200, {"Content-Type": "text/plain"});
    response.write("Hello World");
    response.end();
  }

  /* ... */
}
\end{verbatim}
\end{frame}

\begin{frame}[fragile]\frametitle{Final Piece}

Importing our router and passing it into our HTTP Server:

\begin{verbatim}
var server = require("./server");
var router = require("./router");

server.start(router.route);
\end{verbatim}
Now we have a basic router. It doesn't do anything but now we can write
our request handlers.

\end{frame}

\section{Request Handlers}

\begin{frame}[fragile]\frametitle{Basic Request Handlers}

\begin{verbatim}
function start() {
  console.log("Request handler 'start' was called.");
}

function upload() {
  console.log("Request handler 'upload' was called.");
}

exports.start = start;
exports.upload = upload;
\end{verbatim}
requestHandlers.js

\end{frame}

\begin{frame}\frametitle{Wire This Up in App.js}

var server = require(``./server''); var router = require(``./router'');
var requestHandlers = require(``./requestHandlers'');

var handle = \{\} handle{[}``/''{]} = requestHandlers.start;
handle{[}``/start''{]} = requestHandlers.start; handle{[}``/upload''{]}
= requestHandlers.upload;

server.start(router.route, handle);

We create a dictionary where we defined what paths use what handlers.

\end{frame}

\section{Conclusion}

\begin{frame}\frametitle{Questions?}

\end{frame}

\begin{frame}\frametitle{Links}

\begin{itemize}
\item
  Presentation based on \href{http://www.nodebeginner.org/}{The Node
  Beginner Book}
\item
  \href{http://debuggable.com/posts/understanding-node-js:4bd98440-45e4-4a9a-8ef7-0f7ecbdd56cb}{Understanding
  node.js}
\item
  \href{http://nodejs.org}{Node.JS Website}
\item
  \href{http://github.com/ManuelKiessling/NodeBeginnerBook/tree/master/code/application}{Complete
  Source Code for Application}
\end{itemize}
\end{frame}

\end{document}
